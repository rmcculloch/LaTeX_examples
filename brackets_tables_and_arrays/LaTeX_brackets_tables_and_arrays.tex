%This .tex document demonstrates how to render brackets, tables and arrays.
\documentclass{article}

\begin{document}

$$(x+1)$$
$$3[2+(x+1)]$$
%Note that the { needs to be escaped with a backslash for it to render.
$$\{a,b,c\}$$
%This is the same for the dollar sign which is also a reserved symbol in LaTeX.
$$\$12.55$$
%Note the backslash left and right to ensure the brackets reach the full height
%of the fraction.
$$3\left(\frac{2}{5}\right)$$
$$3\left[\frac{2}{5}\right]$$
$$3\left\{\frac{2}{5}\right\}$$

$$\left|\frac{x}{x+1}\right|$$
%Note when you don't want a closing bracket, LaTeX still wants to see the \right
%command. You just need to replace the bracket with a dot so that nothing
%renders.
$$\left\{x^2\right.$$
%Works the same for opening brackets
$$\left. \frac{dy}{dx} \right|_{x=1}$$

%Note that the c's following \begin{tabular} tell LaTeX to center these columns.
%The | between the c's tell LaTeX to render vertical lines between the columns.
%The \hlines insert horizontal lines to complete the bordered table.
\begin{tabular}{|c|c|c|c|c|c|}
\hline
$x$ & 1 & 2 & 3 & 4 & 5\\ \hline
$f(x)$ & 10 & 11 & 12 & 13 & 14 \\ \hline
\end{tabular}

%Note that the * at the end of \begin{eqnarray*} and \end{eqnarray*} prevent the
%numbering from being rendered.
%Note that the & on either side of the = and \approx tell LaTeX to line up the
%equations at these points.
\begin{eqnarray*}
5x^2-9&=&x+3\\
4x^2&=&12\\
x^3&=&3\\
x&\approx&\pm1.732
\end{eqnarray*}


\end{document}
