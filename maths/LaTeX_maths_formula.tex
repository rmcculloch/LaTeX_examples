\documentclass[11pt]{article}

%Note that additional packages are not required to display the following maths
%symbols/notation

%\usepackage{amsmath}
%\usepackage{amssymb}
%\usepackage{graphicx}

\begin{document}

%Example of math mode
Here is some basic inline maths notation of $(x+1)$ and $(x+3)$.\\
And here is an equation in maths display mode: $$A=x^2+4x+3$$

Superscripts:
$$2x^3$$
%Note that curly brackets required when you have more than just a single
%character exponent
$$2x^{34}$$
$$2x^{3x+4}$$
$$2x^{3x^4+5}$$

Subscripts:
$$x_1$$
$$x_{12}$$
%Note that use of curly brackets allow multiple subscripting
$${{x_1}_2}_3$$

Greek letters:
$$\pi$$
$$\alpha$$
$$A=\pi r^2$$

Log functions:
$$\log_5{x}$$
$$\ln{x}$$

Square roots:
$$\sqrt{2}$$
%Note the square brackets define the nth root. If left out it is assumed to be
%the square root. In the case below it is the cubed root.
$$\sqrt[3]{2}$$
$$\sqrt{x^2+y^2}$$
$$\sqrt{1+\sqrt{x}}$$

Fractions:

About $\frac{2}{3}$ of the glass is full.

%Note the \displaystyle command increases the size of the fraction if it needs
%to be easier to read
About $\displaystyle{\frac{2}{3}}$ of the glass is full.
$$\frac{x}{x^2+x+1}$$
$$\frac{\sqrt{x+1}}{\sqrt{x-1}}$$
$$\frac{1}{1+\frac{1}{x}}$$
$$\sqrt{\frac{x}{x^2+x+1}}$$

\end{document}
