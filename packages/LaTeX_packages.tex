%This .tex document just lists a number of packages and what they do. It will
%not compile as is.

%Including this package will automatically reformat to use more of the page.
\usepackage{fullpage}

%This is a different package allowing you to reformat the page. Allows you to
%enter specific settings for the margins. Could also use centimeters (cm).
\usepackage[top=1in, bottom=1in, left=1in, right=1in]{geometry}

%Another way to format with the geometry package using papersize dimensions.
\usepackage[margin=1in, paperwidth=8.5in, paperheight=11in]{geometry}

%This package allows the use of Rho and other maths symbols
\usepackage{amssymb}

%This package is for the numbering of lines. Useful for sending to people who
%don't use LaTeX to edit a file.
\usepackage{lineno}

%This package is for landscaping pages in a document.
\usepackage{lscape}

%This package is the defacto bibliography package in LaTeX. Biblatex is the
%other package but it may still be in a state of development and not very
%reliable.
\usepackage{natlib}

%This package allows for the use of subfigures.
\usepackage{subfig}

%This package provides hyperreferencing within the pdf and the same with
%citations linking to its reference.
\usepackage[colorlinks=true]{hyperref}

%This package brings in specialized symbols used in mathematics.
\usepackage{amsfonts}

%This package allows you to insert lorem ipsum text
\usepackage[lipsum]

%This package allows for the insertion of images files into our document. It
%accepts only .png, .jpg, .gif and .pdf.
\usepackage{graphicx}
